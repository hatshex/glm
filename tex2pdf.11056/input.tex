\documentclass[]{article}
\usepackage{lmodern}
\usepackage{amssymb,amsmath}
\usepackage{ifxetex,ifluatex}
\usepackage{fixltx2e} % provides \textsubscript
\ifnum 0\ifxetex 1\fi\ifluatex 1\fi=0 % if pdftex
  \usepackage[T1]{fontenc}
  \usepackage[utf8]{inputenc}
\else % if luatex or xelatex
  \ifxetex
    \usepackage{mathspec}
  \else
    \usepackage{fontspec}
  \fi
  \defaultfontfeatures{Ligatures=TeX,Scale=MatchLowercase}
\fi
% use upquote if available, for straight quotes in verbatim environments
\IfFileExists{upquote.sty}{\usepackage{upquote}}{}
% use microtype if available
\IfFileExists{microtype.sty}{%
\usepackage{microtype}
\UseMicrotypeSet[protrusion]{basicmath} % disable protrusion for tt fonts
}{}
\usepackage[margin=1in]{geometry}
\usepackage{hyperref}
\hypersetup{unicode=true,
            pdftitle={Ejercicio 2},
            pdfborder={0 0 0},
            breaklinks=true}
\urlstyle{same}  % don't use monospace font for urls
\usepackage{color}
\usepackage{fancyvrb}
\newcommand{\VerbBar}{|}
\newcommand{\VERB}{\Verb[commandchars=\\\{\}]}
\DefineVerbatimEnvironment{Highlighting}{Verbatim}{commandchars=\\\{\}}
% Add ',fontsize=\small' for more characters per line
\usepackage{framed}
\definecolor{shadecolor}{RGB}{248,248,248}
\newenvironment{Shaded}{\begin{snugshade}}{\end{snugshade}}
\newcommand{\KeywordTok}[1]{\textcolor[rgb]{0.13,0.29,0.53}{\textbf{{#1}}}}
\newcommand{\DataTypeTok}[1]{\textcolor[rgb]{0.13,0.29,0.53}{{#1}}}
\newcommand{\DecValTok}[1]{\textcolor[rgb]{0.00,0.00,0.81}{{#1}}}
\newcommand{\BaseNTok}[1]{\textcolor[rgb]{0.00,0.00,0.81}{{#1}}}
\newcommand{\FloatTok}[1]{\textcolor[rgb]{0.00,0.00,0.81}{{#1}}}
\newcommand{\ConstantTok}[1]{\textcolor[rgb]{0.00,0.00,0.00}{{#1}}}
\newcommand{\CharTok}[1]{\textcolor[rgb]{0.31,0.60,0.02}{{#1}}}
\newcommand{\SpecialCharTok}[1]{\textcolor[rgb]{0.00,0.00,0.00}{{#1}}}
\newcommand{\StringTok}[1]{\textcolor[rgb]{0.31,0.60,0.02}{{#1}}}
\newcommand{\VerbatimStringTok}[1]{\textcolor[rgb]{0.31,0.60,0.02}{{#1}}}
\newcommand{\SpecialStringTok}[1]{\textcolor[rgb]{0.31,0.60,0.02}{{#1}}}
\newcommand{\ImportTok}[1]{{#1}}
\newcommand{\CommentTok}[1]{\textcolor[rgb]{0.56,0.35,0.01}{\textit{{#1}}}}
\newcommand{\DocumentationTok}[1]{\textcolor[rgb]{0.56,0.35,0.01}{\textbf{\textit{{#1}}}}}
\newcommand{\AnnotationTok}[1]{\textcolor[rgb]{0.56,0.35,0.01}{\textbf{\textit{{#1}}}}}
\newcommand{\CommentVarTok}[1]{\textcolor[rgb]{0.56,0.35,0.01}{\textbf{\textit{{#1}}}}}
\newcommand{\OtherTok}[1]{\textcolor[rgb]{0.56,0.35,0.01}{{#1}}}
\newcommand{\FunctionTok}[1]{\textcolor[rgb]{0.00,0.00,0.00}{{#1}}}
\newcommand{\VariableTok}[1]{\textcolor[rgb]{0.00,0.00,0.00}{{#1}}}
\newcommand{\ControlFlowTok}[1]{\textcolor[rgb]{0.13,0.29,0.53}{\textbf{{#1}}}}
\newcommand{\OperatorTok}[1]{\textcolor[rgb]{0.81,0.36,0.00}{\textbf{{#1}}}}
\newcommand{\BuiltInTok}[1]{{#1}}
\newcommand{\ExtensionTok}[1]{{#1}}
\newcommand{\PreprocessorTok}[1]{\textcolor[rgb]{0.56,0.35,0.01}{\textit{{#1}}}}
\newcommand{\AttributeTok}[1]{\textcolor[rgb]{0.77,0.63,0.00}{{#1}}}
\newcommand{\RegionMarkerTok}[1]{{#1}}
\newcommand{\InformationTok}[1]{\textcolor[rgb]{0.56,0.35,0.01}{\textbf{\textit{{#1}}}}}
\newcommand{\WarningTok}[1]{\textcolor[rgb]{0.56,0.35,0.01}{\textbf{\textit{{#1}}}}}
\newcommand{\AlertTok}[1]{\textcolor[rgb]{0.94,0.16,0.16}{{#1}}}
\newcommand{\ErrorTok}[1]{\textcolor[rgb]{0.64,0.00,0.00}{\textbf{{#1}}}}
\newcommand{\NormalTok}[1]{{#1}}
\usepackage{graphicx,grffile}
\makeatletter
\def\maxwidth{\ifdim\Gin@nat@width>\linewidth\linewidth\else\Gin@nat@width\fi}
\def\maxheight{\ifdim\Gin@nat@height>\textheight\textheight\else\Gin@nat@height\fi}
\makeatother
% Scale images if necessary, so that they will not overflow the page
% margins by default, and it is still possible to overwrite the defaults
% using explicit options in \includegraphics[width, height, ...]{}
\setkeys{Gin}{width=\maxwidth,height=\maxheight,keepaspectratio}
\IfFileExists{parskip.sty}{%
\usepackage{parskip}
}{% else
\setlength{\parindent}{0pt}
\setlength{\parskip}{6pt plus 2pt minus 1pt}
}
\setlength{\emergencystretch}{3em}  % prevent overfull lines
\providecommand{\tightlist}{%
  \setlength{\itemsep}{0pt}\setlength{\parskip}{0pt}}
\setcounter{secnumdepth}{0}
% Redefines (sub)paragraphs to behave more like sections
\ifx\paragraph\undefined\else
\let\oldparagraph\paragraph
\renewcommand{\paragraph}[1]{\oldparagraph{#1}\mbox{}}
\fi
\ifx\subparagraph\undefined\else
\let\oldsubparagraph\subparagraph
\renewcommand{\subparagraph}[1]{\oldsubparagraph{#1}\mbox{}}
\fi

%%% Use protect on footnotes to avoid problems with footnotes in titles
\let\rmarkdownfootnote\footnote%
\def\footnote{\protect\rmarkdownfootnote}

%%% Change title format to be more compact
\usepackage{titling}

% Create subtitle command for use in maketitle
\newcommand{\subtitle}[1]{
  \posttitle{
    \begin{center}\large#1\end{center}
    }
}

\setlength{\droptitle}{-2em}
  \title{Ejercicio 2}
  \pretitle{\vspace{\droptitle}\centering\huge}
  \posttitle{\par}
  \author{}
  \preauthor{}\postauthor{}
  \predate{\centering\large\emph}
  \postdate{\par}
  \date{19 de septiembre de 2016}


\begin{document}
\maketitle

\begin{enumerate}
\def\labelenumi{\arabic{enumi})}
\setcounter{enumi}{1}
\tightlist
\item
  Las utilidades mensuales de una compañía tienen una distribución
  N(\(\mu, \sigma^2\)). Suponga que una muestra de 10 meses de esta
  compañía dio como resultado las siguientes utilidades: (212, 207, 210,
  196, 223, 193, 196, 210, 202, 221).
\end{enumerate}

\begin{itemize}
\item
  \begin{enumerate}
  \def\labelenumi{\alph{enumi})}
  \tightlist
  \item
    La incertidumbre sobre la utilidad promedio anual \(\mu\) se puede
    representar por una distribución N(200,40), y la incertidumbre de la
    desviación estándar de las utilidades mensuales se puede representar
    mediante una distribución Ga(10,1). Mediante la distribución
    posterior estima \(\mu\) y \(\sigma^2\).
  \end{enumerate}
\item
  \begin{enumerate}
  \def\labelenumi{\alph{enumi})}
  \setcounter{enumi}{1}
  \tightlist
  \item
    Utilizando una distribución inicial no informativa, estima mediante
    la correspondiente distribución inicial \(\mu\) y \(\sigma^2\) .
  \end{enumerate}
\end{itemize}

\subsubsection{Definimos el modelo}\label{definimos-el-modelo}

\begin{itemize}
\item
  Las variables son de tipo Normal (\(\mu, \tau\))
\item
  f(\(\mu, \sigma^2\)) es nuestra distribución inicial
\item
  Utilizamos el archivo \textbf{Eje2\_a.txt} para describir nuestro
  modelo:

\begin{verbatim}
model
{
#Likelihood
for (i in 1:n) {
x[i] ~ dnorm(mu,tau)
}
tau<-1/pow(sig,2)
#Priors 
#----------
#Inciso a
mu ~ dnorm(200,tau0)
tau0<-1/40
sig ~ dgamma(10,1)
#----------
#Inciso b
#mu ~ dnorm(0,0.0001)
#sig ~ dgamma(0.001,0.001)
#Prediction
x1 ~ dnorm(mu,tau)
}
\end{verbatim}
\end{itemize}

\subsubsection{Definimos los datos, inicializamos y seleccionamos los
parámetros a
monitorear.}\label{definimos-los-datos-inicializamos-y-seleccionamos-los-parametros-a-monitorear.}

\begin{Shaded}
\begin{Highlighting}[]
\KeywordTok{library}\NormalTok{(R2OpenBUGS)}
\KeywordTok{library}\NormalTok{(R2jags)}
\KeywordTok{library}\NormalTok{(ggplot2)}
\end{Highlighting}
\end{Shaded}

\begin{Shaded}
\begin{Highlighting}[]
\CommentTok{#-Working directory-}
\KeywordTok{setwd}\NormalTok{(}\StringTok{"E:/itam/2016 Verano/glm/"}\NormalTok{)}

\CommentTok{#--- Ejemplo 2---}
\CommentTok{#--Datos}
\NormalTok{utilidad<-}\KeywordTok{c}\NormalTok{(}\DecValTok{212}\NormalTok{, }\DecValTok{207}\NormalTok{, }\DecValTok{210}\NormalTok{, }\DecValTok{196}\NormalTok{, }\DecValTok{223}\NormalTok{, }\DecValTok{193}\NormalTok{, }\DecValTok{196}\NormalTok{, }\DecValTok{210}\NormalTok{, }\DecValTok{202}\NormalTok{, }\DecValTok{221}\NormalTok{)}
\NormalTok{n<-}\KeywordTok{length}\NormalTok{(utilidad)}

\CommentTok{#-Defining data-}
\NormalTok{data<-}\KeywordTok{list}\NormalTok{(}\StringTok{"n"}\NormalTok{=n,}\StringTok{"x"}\NormalTok{=utilidad)}

\CommentTok{#-Defining inits-}
\NormalTok{inits<-function()\{}\KeywordTok{list}\NormalTok{(}\DataTypeTok{mu=}\DecValTok{0}\NormalTok{,}\DataTypeTok{sig=}\DecValTok{1}\NormalTok{,}\DataTypeTok{x1=}\DecValTok{0}\NormalTok{)\}}

\CommentTok{#-Selecting parameters to monitor-}
\NormalTok{parameters<-}\KeywordTok{c}\NormalTok{(}\StringTok{"mu"}\NormalTok{,}\StringTok{"sig"}\NormalTok{,}\StringTok{"x1"}\NormalTok{)}
\end{Highlighting}
\end{Shaded}

\subsubsection{Simulamos con el modelo previamente definido y
graficamos.}\label{simulamos-con-el-modelo-previamente-definido-y-graficamos.}

\begin{Shaded}
\begin{Highlighting}[]
\CommentTok{#-Running code-}
\CommentTok{#OpenBUGS}
\NormalTok{ej2.sim<-}\KeywordTok{bugs}\NormalTok{(data,inits,parameters,}\DataTypeTok{model.file=}\StringTok{"Ej2_a.txt"}\NormalTok{,           }\DataTypeTok{n.iter=}\DecValTok{5000}\NormalTok{,}\DataTypeTok{n.chains=}\DecValTok{1}\NormalTok{,}\DataTypeTok{n.burnin=}\DecValTok{500}\NormalTok{)}

\CommentTok{#-Monitoring chain-}

\CommentTok{#Traza de la cadena}
\KeywordTok{traceplot}\NormalTok{(ej2.sim)}
\end{Highlighting}
\end{Shaded}

\includegraphics{ejercicio02_files/figure-latex/unnamed-chunk-3-1.pdf}
\includegraphics{ejercicio02_files/figure-latex/unnamed-chunk-3-2.pdf}
\includegraphics{ejercicio02_files/figure-latex/unnamed-chunk-3-3.pdf}
\includegraphics{ejercicio02_files/figure-latex/unnamed-chunk-3-4.pdf}

\begin{Shaded}
\begin{Highlighting}[]
\KeywordTok{dev.off}\NormalTok{()}
\end{Highlighting}
\end{Shaded}

\begin{verbatim}
## null device 
##           1
\end{verbatim}

\begin{Shaded}
\begin{Highlighting}[]
\CommentTok{#Cadena}
\CommentTok{#OpenBUGS}
\NormalTok{out<-ej2.sim$sims.list}

\NormalTok{z<-out$x1}
\KeywordTok{par}\NormalTok{(}\DataTypeTok{mfrow=}\KeywordTok{c}\NormalTok{(}\DecValTok{2}\NormalTok{,}\DecValTok{2}\NormalTok{))}
\KeywordTok{plot}\NormalTok{(z,}\DataTypeTok{type=}\StringTok{"l"}\NormalTok{)}
\KeywordTok{plot}\NormalTok{(}\KeywordTok{cumsum}\NormalTok{(z)/(}\DecValTok{1}\NormalTok{:}\KeywordTok{length}\NormalTok{(z)),}\DataTypeTok{type=}\StringTok{"l"}\NormalTok{)}
\KeywordTok{hist}\NormalTok{(z,}\DataTypeTok{freq=}\OtherTok{FALSE}\NormalTok{)}
\KeywordTok{acf}\NormalTok{(z)}
\end{Highlighting}
\end{Shaded}

\includegraphics{ejercicio02_files/figure-latex/unnamed-chunk-4-1.pdf}

\begin{Shaded}
\begin{Highlighting}[]
\CommentTok{#Resumen (estimadores)}
\CommentTok{#OpenBUGS}
\NormalTok{out.sum<-ej2.sim$summary}
\KeywordTok{print}\NormalTok{(out.sum)}
\end{Highlighting}
\end{Shaded}

\begin{verbatim}
##               mean        sd      2.5%      25%     50%      75%     97.5%
## mu       205.47244  2.977070 199.30000 203.6000 205.600 207.4000 211.00000
## sig       10.54300  1.939524   7.42795   9.1480  10.295  11.6625  15.01525
## x1       205.33918 11.240333 183.10000 198.1000 205.300 212.6000 227.70000
## deviance  75.62381  1.650611  74.03000  74.4575  75.130  76.2700  79.97149
\end{verbatim}

\begin{Shaded}
\begin{Highlighting}[]
\CommentTok{#DIC}
\CommentTok{#OpenBUGS}
\NormalTok{out.dic<-ej2.sim$DIC}
\KeywordTok{print}\NormalTok{(out.dic)}
\end{Highlighting}
\end{Shaded}

\begin{verbatim}
## [1] 76.98607
\end{verbatim}

\subsubsection{Ahora para el inciso b, utilizando una distribución
inicial no
informativa.}\label{ahora-para-el-inciso-b-utilizando-una-distribucion-inicial-no-informativa.}

\begin{itemize}
\item
  Utilizamos el archivo \textbf{Eje2\_b.txt} para describir nuestro
  modelo:

\begin{verbatim}
model
{
#Likelihood
for (i in 1:n) {
x[i] ~ dnorm(mu,tau)
}
tau<-1/pow(sig,2)
#Priors 
#----------
#Inciso b
mu ~ dnorm(0,0.0001)
sig ~ dgamma(0.001,0.001)
#Prediction
x1 ~ dnorm(mu,tau)
}
\end{verbatim}
\end{itemize}

\subsubsection{Simulamos con el modelo previamente definido y
graficamos.}\label{simulamos-con-el-modelo-previamente-definido-y-graficamos.-1}

\begin{Shaded}
\begin{Highlighting}[]
\CommentTok{#-Running code-}
\CommentTok{#OpenBUGS}
\NormalTok{ej2.sim<-}\KeywordTok{bugs}\NormalTok{(data,inits,parameters,}\DataTypeTok{model.file=}\StringTok{"Ej2_b.txt"}\NormalTok{,           }\DataTypeTok{n.iter=}\DecValTok{5000}\NormalTok{,}\DataTypeTok{n.chains=}\DecValTok{1}\NormalTok{,}\DataTypeTok{n.burnin=}\DecValTok{500}\NormalTok{)}

\CommentTok{#-Monitoring chain-}

\CommentTok{#Traza de la cadena}
\KeywordTok{traceplot}\NormalTok{(ej2.sim)}
\end{Highlighting}
\end{Shaded}

\includegraphics{ejercicio02_files/figure-latex/unnamed-chunk-7-1.pdf}
\includegraphics{ejercicio02_files/figure-latex/unnamed-chunk-7-2.pdf}
\includegraphics{ejercicio02_files/figure-latex/unnamed-chunk-7-3.pdf}
\includegraphics{ejercicio02_files/figure-latex/unnamed-chunk-7-4.pdf}

\begin{Shaded}
\begin{Highlighting}[]
\KeywordTok{dev.off}\NormalTok{()}
\end{Highlighting}
\end{Shaded}

\begin{verbatim}
## null device 
##           1
\end{verbatim}

\begin{Shaded}
\begin{Highlighting}[]
\CommentTok{#Cadena}
\CommentTok{#OpenBUGS}
\NormalTok{out<-ej2.sim$sims.list}

\NormalTok{z<-out$x1}
\KeywordTok{par}\NormalTok{(}\DataTypeTok{mfrow=}\KeywordTok{c}\NormalTok{(}\DecValTok{2}\NormalTok{,}\DecValTok{2}\NormalTok{))}
\KeywordTok{plot}\NormalTok{(z,}\DataTypeTok{type=}\StringTok{"l"}\NormalTok{)}
\KeywordTok{plot}\NormalTok{(}\KeywordTok{cumsum}\NormalTok{(z)/(}\DecValTok{1}\NormalTok{:}\KeywordTok{length}\NormalTok{(z)),}\DataTypeTok{type=}\StringTok{"l"}\NormalTok{)}
\KeywordTok{hist}\NormalTok{(z,}\DataTypeTok{freq=}\OtherTok{FALSE}\NormalTok{)}
\KeywordTok{acf}\NormalTok{(z)}
\end{Highlighting}
\end{Shaded}

\includegraphics{ejercicio02_files/figure-latex/unnamed-chunk-8-1.pdf}

\begin{Shaded}
\begin{Highlighting}[]
\CommentTok{#Resumen (estimadores)}
\CommentTok{#OpenBUGS}
\NormalTok{out.sum<-ej2.sim$summary}
\KeywordTok{print}\NormalTok{(out.sum)}
\end{Highlighting}
\end{Shaded}

\begin{verbatim}
##               mean        sd       2.5%      25%    50%     75%     97.5%
## mu       206.70280  3.658073 199.000000 204.5000 206.80 208.900 213.90000
## sig       11.22055  2.960489   7.102596   9.1395  10.67  12.650  18.53050
## x1       206.43924 12.116687 182.100000 199.2000 206.30 214.025 230.40000
## deviance  76.16268  2.144395  74.060000  74.6200  75.49  77.000  82.07525
\end{verbatim}

\begin{Shaded}
\begin{Highlighting}[]
\CommentTok{#DIC}
\CommentTok{#OpenBUGS}
\NormalTok{out.dic<-ej2.sim$DIC}
\KeywordTok{print}\NormalTok{(out.dic)}
\end{Highlighting}
\end{Shaded}

\begin{verbatim}
## [1] 78.46189
\end{verbatim}


\end{document}
